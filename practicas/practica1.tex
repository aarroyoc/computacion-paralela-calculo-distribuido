% Created 2024-04-07 dom 22:33
% Intended LaTeX compiler: pdflatex
\documentclass[a4paper]{article}
\usepackage[utf8]{inputenc}
\usepackage[T1]{fontenc}
\usepackage{graphicx}
\usepackage{longtable}
\usepackage{wrapfig}
\usepackage{rotating}
\usepackage[normalem]{ulem}
\usepackage{amsmath}
\usepackage{amssymb}
\usepackage{capt-of}
\usepackage{hyperref}
\usepackage[spanish, american]{babel}
\author{Adrián Arroyo Calle}
\date{Curso 2023-2024}
\title{Práctica 1 - GEMM}
\hypersetup{
 pdfauthor={Adrián Arroyo Calle},
 pdftitle={Práctica 1 - GEMM},
 pdfkeywords={},
 pdfsubject={},
 pdfcreator={Emacs 29.3 (Org mode 9.6.15)}, 
 pdflang={English}}
\begin{document}

\maketitle

\section{Objetivo}
\label{sec:org3bcb720}

Desarrollar una implementación paralela eficiente del producto matriz-matriz generalizado (GEMM) usando Julia.

Se define la operación GEMM como:

$$
C = C + AB
$$

donde \(C \in \mathbb{R}^{m \times n}\), \(A \in \mathbb{R}^{m \times k}\), \(B \in \mathbb{R}^{k \times n}\) son matrices con valores reales.

Para ello deberá implementarse:

\subsection{GEMM secuencial}
\label{sec:org3294a45}

Un GEMM secuencial basado en tres bucles.

\begin{verbatim}
procedure GEMM (A, B, C)
  input: A M*K-matrix, B K*N-matrix, C M*K-matrix
  output: C M*N-matrix

  for ( i = 0; i < M; i++ )
    for ( j = 0; j < N; j++ )
      for ( k = 0; k < K; k++ )
        C[i,j] = C[i,j] + A[i,k] * B[k,j]
end GEMM
\end{verbatim}

\subsection{Validador}
\label{sec:orgb603aef}

Simplemente ejecuta una GEMM secuencial sobre las matrices A, B, C dadas
y comprueba que el resultado coincida con el almacenado en otra matriz resultado C'
dada. Esta comprobación debe hacerse sumando las diferencias entre ambas en valor absoluto:

$$
\sum_{i=1}^{m}\sum_{j=1}^{n} \mid C_{i,j} - C'_{i_j} \mid
$$

El validador se utilizará para verificar el resultado
de las ejecuciones paralelas.

\subsection{GEMM paralela}
\label{sec:orga14d7f1}

En la que el trabajo de uno (o varios) bucles se reparta entre los hilos
que intervienen en la ejecución.

\section{Trabajo a realizar}
\label{sec:orgc8dce35}

\begin{enumerate}
\item Implementar la GEMM secuencial y el validador descritos en la sección anterior.
\item Implementar 3 versiones válidas (el resultado es numéricamente correcto) de la GEMM paralelizada.
\item Justificar un caso en el que un determinado bucle de la GEMM no pueda ser paralelizado (el resultado sería numericamente incorrecto).
\item Realizar un análisis de escalabilidad (fuerte o débil) describiendo las dimensiones de
las matrices utilizadas en dicho análisis, así como la cantidad de hilos utilizados en las
pruebas y el número de repeticiones realizadas de cada ejecución. Se presentará tanto
una tabla reflejando los tiempos de ejecución medidos (el asociado a cada repetición
y la media), como una gráfica que ilustre la escalabilidad que se observa.

\textbf{Observación 1}: Se recomienda, al menos, repetir cada medida cinco veces (anotando
todos los tiempos obtenidos en la tabla).

\textbf{Observación 2}: En el caso de que se detecte algún outlier, debe eliminarse antes de hacer
el cálculo de la media, marcándolo como outlier en la tabla.

\textbf{Observación 3}: Basta con medir el tiempo de la región paralela (tp), no es necesario medir
también el tiempo total de ejecución.

\textbf{Observación 4}: Las matrices utilizadas en el proceso de experimentación deben ser su-
ficientemente grandes como para que se llegue a observar una mejora del tiempo de
ejecución al usar la implementación paralela (no hay que tener miedo a probar productos
de matrices 1000x1000, 2000x2000\ldots{}). Tras validar con matrices pequeñas (100x100, por
ejemplo) que el resultado derivado de la ejecución paralela es correcto, puede "comen-
tarse" la llamada al validador en el código para no "perder el tiempo" con comprobaciones
innecesarias.
\item Incluir en la tabla realizada en el apartado anterior una columna donde se indiquen el
\textbf{speedup} y la \textbf{eficiencia} (según la media de tiempo calculada).

La tabla podría acabar siendo algo así:

\begin{center}
\begin{tabular}{rllllllll}
Cantidad de hilos & TP1 & TP2 & TP3 & TP4 & TP5 & MEDIA & Speedup & Eficiencia\\[0pt]
\hline
1 (secuencial) &  &  &  &  &  &  &  & \\[0pt]
2 &  &  &  &  &  &  &  & \\[0pt]
4 &  &  &  &  &  &  &  & \\[0pt]
6 &  &  &  &  &  &  &  & \\[0pt]
\ldots{} &  &  &  &  &  &  &  & \\[0pt]
\end{tabular}
\end{center}
\item Escribir una pequeña reflexión a modo de conclusión del trabajo realizado (uno o dos
párrafos).
\end{enumerate}

\section{Entrega}
\label{sec:org75c2111}

La práctica deberá entregarse antes de TODO

Se deberán presentar las distintas implementaciones (en forma de código Julia con extensión .jl) y
un documento breve incluyendo los resultados derivados de la evaluación (tabla y gráficas), así como
las reflexiones finales.
\end{document}
