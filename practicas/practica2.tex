% Created 2024-04-07 dom 22:33
% Intended LaTeX compiler: pdflatex
\documentclass[a4paper]{article}
\usepackage[utf8]{inputenc}
\usepackage[T1]{fontenc}
\usepackage{graphicx}
\usepackage{longtable}
\usepackage{wrapfig}
\usepackage{rotating}
\usepackage[normalem]{ulem}
\usepackage{amsmath}
\usepackage{amssymb}
\usepackage{capt-of}
\usepackage{hyperref}
\usepackage[spanish, american]{babel}
\author{Adrián Arroyo Calle}
\date{Curso 2023-2024}
\title{Práctica 2 - GEMM}
\hypersetup{
 pdfauthor={Adrián Arroyo Calle},
 pdftitle={Práctica 2 - GEMM},
 pdfkeywords={},
 pdfsubject={},
 pdfcreator={Emacs 29.3 (Org mode 9.6.15)}, 
 pdflang={English}}
\begin{document}

\maketitle

\section{Objetivo}
\label{sec:org4f31ce4}

Desarrollar una implementación paralela eficiente del producto matriz-matriz generalizado (GEMM) usando Julia.

Se define la operación GEMM como:

$$
C = C + AB
$$

donde \(C \in \mathbb{R}^{m \times n}\), \(A \in \mathbb{R}^{m \times k}\), \(B \in \mathbb{R}^{k \times n}\) son matrices con valores reales.

Para ello deberá implementarse:

\subsection{GEMM secuencial}
\label{sec:orgc6eb738}

Un GEMM secuencial basado en tres bucles.

\begin{verbatim}
procedure GEMM (A, B, C)
  input: A M*K-matrix, B K*N-matrix, C M*K-matrix
  output: C M*N-matrix

  for ( i = 0; i < M; i++ )
    for ( j = 0; j < N; j++ )
      for ( k = 0; k < K; k++ )
        C[i,j] = C[i,j] + A[i,k] * B[k,j]
end GEMM
\end{verbatim}

\subsection{Validador}
\label{sec:org3155083}

Simplemente ejecuta una GEMM secuencial sobre las matrices A, B, C dadas
y comprueba que el resultado coincida con el almacenado en otra matriz resultado C'
dada. Esta comprobación debe hacerse sumando las diferencias entre ambas en valor absoluto:

$$
\sum_{i=1}^{m}\sum_{j=1}^{n} \mid C_{i,j} - C'_{i_j} \mid
$$

El validador se utilizará para verificar el resultado
de las ejecuciones paralelas.

\subsection{GEMM paralela}
\label{sec:org04fba3e}

En la que el trabajo de uno (o varios) bucles se reparta entre diferentes nodos usando técnicas
de programación distribuida.

\section{Trabajo a realizar}
\label{sec:org201142e}

\begin{enumerate}
\item Implementar la GEMM secuencial y el validador descritos en la sección anterior.
\item Implementar una versión de la GEMM paralelizada usando \textbf{DistributedArrays}.
\item Complementar la versión GEMM paralelizada con técnicas de paralelismo en memoria compartida.
\item Si disponemos de máquinas virtuales, hacer un \textbf{análisis de rendimiento} comparando la versión secuencial,
la versión paralelizada solamente con \textbf{DistributedArrays} y la versión que combina paralelismo de memoria
compartida con el de memoria distribuida.
\end{enumerate}
\end{document}
