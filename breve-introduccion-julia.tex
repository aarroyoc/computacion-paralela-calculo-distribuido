% Created 2024-03-03 dom 18:04
% Intended LaTeX compiler: pdflatex
\documentclass[bigger]{beamer}
\usepackage[utf8]{inputenc}
\usepackage[T1]{fontenc}
\usepackage{graphicx}
\usepackage{longtable}
\usepackage{wrapfig}
\usepackage{rotating}
\usepackage[normalem]{ulem}
\usepackage{amsmath}
\usepackage{amssymb}
\usepackage{capt-of}
\usepackage{hyperref}
\mode<beamer>{\usetheme{Madrid}}
\usetheme{default}
\author{Adrián Arroyo Calle}
\date{Curso 2023-2024}
\title{Breve introducción a Julia}
\hypersetup{
 pdfauthor={Adrián Arroyo Calle},
 pdftitle={Breve introducción a Julia},
 pdfkeywords={},
 pdfsubject={},
 pdfcreator={Emacs 29.2 (Org mode 9.6.15)}, 
 pdflang={Spanish}}
\begin{document}

\maketitle

\section{Breve introducción a Julia}
\label{sec:orgce1b015}

\begin{frame}[label={sec:org384dd24}]{Julia}
\begin{center}
\includegraphics[width=0.4\textwidth]{./julia.png}
\end{center}

\begin{itemize}
\item Lenguaje de alto rendimiento para aplicaciones científicas.
\item Sintaxis y semántica similar a Python.
\item Pero con rendimiento más cercano a C o FORTRAN.
\item Relativamente reciente
\end{itemize}
\end{frame}

\begin{frame}[label={sec:orgdafe723}]{Descargar Julia}
\url{https://julialang.org/}
\end{frame}

\begin{frame}[label={sec:orgf0dd4a9},fragile]{Ejecutar programas}
 \begin{block}{Ejecutar programa en Julia}
\begin{verbatim}
$ julia script.jl
\end{verbatim}
\end{block}

\begin{block}{Abrir REPL}
\begin{verbatim}
$ julia
\end{verbatim}
\end{block}
\end{frame}

\begin{frame}[label={sec:org4b8b702},fragile]{Variables}
 \begin{itemize}
\item Asignamos variables con \texttt{=}.
\item Las variables son mutables.
\item Julia es tipado pero existe \alert{inferencia de tipos}.
\end{itemize}
\begin{verbatim}
a = 14
b = false
c = "cadena de texto"
\end{verbatim}
\end{frame}

\begin{frame}[label={sec:orgcb1e97f},fragile]{Vectores y matrices}
 \begin{itemize}
\item Podemos crear vectores y matrices rellenos de ceros con \texttt{zeros}.
\item Vectores de 1D y 2D tienen sintaxis especiales.
\item Los elementos se enumeran empezando por 1 y acabando en \texttt{end}.
\item Podemos obtener las dimensiones con \texttt{size} o \texttt{length}.
\end{itemize}
\end{frame}

\begin{frame}[label={sec:org58032d3},fragile]{Vectores y matrices}
 \begin{verbatim}
a = zeros(2,2)
a[1,1] = 5
a[1,2] = 6
a[end,2] = 8
a[end,end] = 10
julia> a
2×2 Matrix{Float64}:
 5.0   6.0
 0.0  10.0
\end{verbatim}
\end{frame}

\begin{frame}[label={sec:org8a9261a},fragile]{Vectores y matrices}
 \begin{verbatim}
a = [1,2,3,4] # 4-element Vector{Int64}
a = [1 2;3 4] # 2x2 Matrix{Int64}
length(a) # 4
size(a, 1) # 2
size(a, 2) # 2
a[:,1] # 1 3
a[1,:] # 1 2
\end{verbatim}
\end{frame}

\begin{frame}[label={sec:org5a5b8cb},fragile]{Estructuras de control}
 \begin{itemize}
\item Condicionales con if/elseif/else/end
\item Bucles con while/end y for in
\end{itemize}

\begin{verbatim}
a = 18
if a > 18
  println("Mayor de 18")
else
  println("Menor o igual a 18")
end
\end{verbatim}
\end{frame}

\begin{frame}[label={sec:org8136c72},fragile]{Estructuras de control}
 \begin{verbatim}
a = [1, 2, 3]
suma = 0
n = 1
while n <= length(a)
  suma += a[n]
  n += 1
end
\end{verbatim}
\end{frame}

\begin{frame}[label={sec:orgd6ca58b},fragile]{Estructuras de control}
 \begin{verbatim}
a = [1, 2, 3]
suma = 0
for i in a
  suma += i
end
\end{verbatim}
\end{frame}

\begin{frame}[label={sec:org1e45b1c},fragile]{List comprehensions y rangos}
 \begin{itemize}
\item Podemos expresar rangos con \texttt{A:B} donde A es el inicio y B el final.
\item Podemos generar nuevas listas de forma compacta con list comprehensions
\end{itemize}

\begin{verbatim}
[i^2 for i in 1:10 if i % 2 == 0] # [4, 16, 36, 64, 100]
\end{verbatim}
\end{frame}

\begin{frame}[label={sec:org4f3121f},fragile]{Funciones}
 \begin{itemize}
\item Las funciones se definen con la palabra \texttt{function}
\item Opcionalmente se pueden anotar con tipos
\item La última expresión es el valor de return. También se pueden hacer \texttt{return} explícitos.
\end{itemize}

\begin{verbatim}
function suma(a, b)
  a + b
end
\end{verbatim}
\end{frame}

\begin{frame}[label={sec:orgbecbab4},fragile]{Vectorización}
 \begin{itemize}
\item En Julia las funciones soportan \emph{vectorización}.
\item Una función que trabaja con un elemento individual se puede hacer que trabaje con N elementos.
\item Se usa el punto para hacer la transformación.
\end{itemize}

\begin{verbatim}
cos(0) # 1.0
cos.([0, 0.5, pi, 2*pi]) # [1.0, 0.8775, -1.0, 1.0]
[1, 2, 3] .+ 1 # [2, 3, 4]
\end{verbatim}
\end{frame}

\begin{frame}[label={sec:org5166c68},fragile]{Extra}
 \begin{itemize}
\item Algunas funciones están en módulos que hay que importar con \texttt{using}.
\item Podemos generar matrices aleatorias con \texttt{rand}.
\item Si usamos números aleatorios es importante usar una \emph{seed} para tener resultados reproducibles.
\item Podemos importar archivos Julia desde el REPL con \texttt{include}.
\end{itemize}

\begin{verbatim}
include("fichero.jl")
using Random
Random.seed!(0)
a = rand(10, 10) # Matriz aleatoria 10x10
\end{verbatim}
\end{frame}
\end{document}
