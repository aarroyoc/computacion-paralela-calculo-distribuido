% Created 2024-03-09 sáb 14:07
% Intended LaTeX compiler: pdflatex
\documentclass[bigger]{beamer}
\usepackage[utf8]{inputenc}
\usepackage[T1]{fontenc}
\usepackage{graphicx}
\usepackage{longtable}
\usepackage{wrapfig}
\usepackage{rotating}
\usepackage[normalem]{ulem}
\usepackage{amsmath}
\usepackage{amssymb}
\usepackage{capt-of}
\usepackage{hyperref}
\mode<beamer>{\usetheme{Madrid}}
\usetheme{default}
\author{Adrián Arroyo Calle}
\date{Curso 2023-2024}
\title{Memoria Distribuida}
\hypersetup{
 pdfauthor={Adrián Arroyo Calle},
 pdftitle={Memoria Distribuida},
 pdfkeywords={},
 pdfsubject={},
 pdfcreator={Emacs 29.2 (Org mode 9.6.15)}, 
 pdflang={Spanish}}
\begin{document}

\maketitle

\section{Memoria distribuida}
\label{sec:orga1ea5f9}

\begin{frame}[label={sec:orgf5c10c4}]{Memoria distribuida}
\begin{itemize}
\item El paralelismo basado en memoria compartida se basa en la existencia de procesadores multinúcleo.
\item Hay un número máximo de núcleos que podemos tener en un procesador. Muchas tareas pueden requerir todavía más nodos.
\item Pero podemos conectar diferentes ordenadores vía red.
\item Paradigma diferente: cada nodo tiene \alert{su propia memoria}. Toda la comunicación debe realizarse \alert{vía red}.
\end{itemize}
\end{frame}

\begin{frame}[label={sec:org798d328}]{Memoria distribuida}
\begin{itemize}
\item Así es como funcionan los superordenadores, aunque cada nodo sea muy potente, el verdadero impulso es que todos los nodos
pueden trabajar en la misma tarea.
\item Uno de los estándares más usados es \alert{MPI} (\emph{Message Passing Interface}), compatible con varios lenguajes. En Julia disponible bajo la librería \emph{MPI.jl}.
\item En Julia se dispone además de una librería más sencilla, solo compatible con Julia, llamada \emph{Distributed.jl}.
\end{itemize}
\end{frame}

\begin{frame}[label={sec:orgb2bfa37}]{Clúster}
\begin{itemize}
\item Al conjunto de máquinas que trabajan juntas, así como la comunicación entre ellas, se le llama \alert{clúster}.
\item En un \alert{clúster} no todas las máquinas pueden tener la misma potencia. Para evitar desequilibrios,
es necesario hacer un \alert{balanceo de carga}.
\item Este balanceo puede ser estático, en base a datos que sepamos sobre los nodos o dinámico, según el sistema va
viendo con qué velocidad se procesan los datos.
\end{itemize}
\end{frame}

\begin{frame}[label={sec:org48baefd}]{Clústers en Julia}
\begin{itemize}
\item En Julia los clústers se gestionan con \emph{ClusterManager.jl}. Por defecto tendremos dos tipos de clúster: local y SSH.
\item Local usa como nodos procesos extra dentro de la misma máquina.
\begin{itemize}
\item No mejora el rendimiento respecto a la memoria compartida pero es útil para pruebas
\end{itemize}
\item SSH se conecta a máquinas externas a través del protocolo SSH. Las máquinas deberán tener Julia ya instalado.
\end{itemize}
\end{frame}

\begin{frame}[label={sec:org7c60060},fragile]{Clústers en Julia}
 \begin{verbatim}
using Distributed

Distributed.addprocs(4) # añadir 4 nodos locales
Distributed.addprocs(
    [("hostname", 4)],
    dir="/home/ubuntu",
    tunnel=true) # añadir 4 nodos en la máquina hostname
\end{verbatim}

\begin{itemize}
\item Con \texttt{addprocs} añadimos máquinas al clúster. Dentro de cada máquina podemos ejecutar varios nodos.
\end{itemize}
\end{frame}

\begin{frame}[label={sec:org9471869},fragile]{Consultar clúster}
 \begin{verbatim}
Distributed.procs() # listado de pids (process ids)
Distributed.workers() # listado de pids de los workers
\end{verbatim}

\begin{itemize}
\item Habitualmente hay 1 proceso más que el número de workers.
\item Hay un proceso que se dedica a la sincronización. Normalmente el primer proceso.
\end{itemize}
\end{frame}

\begin{frame}[label={sec:org8234781},fragile]{@spawnat}
 \begin{itemize}
\item \texttt{@spawnat} es similar a \texttt{@spawn}, nos permite lanzar una función a otro nodo.
\item Es obligatorio indicar el \emph{PID} del nodo que va a procesar o \texttt{:any}.
\item Al hacer \texttt{@spawnat} se lanza un mensaje a través de la red para que el nodo empiece a trabajar.
\item Más tarde podemos obtener el resultado con \texttt{fetch}, que manda otro mensaje y el nodo responde.
\end{itemize}
\end{frame}

\begin{frame}[label={sec:orgd495179}]{@spawnat}
\begin{itemize}
\item Aquí la memoria no es compartida, todo lo que queramos que se ejecute, se lo debemos pasar.
\item Incluidas las funciones.
\end{itemize}
\end{frame}

\begin{frame}[label={sec:org1a11039},fragile]{@spawnat}
 \begin{verbatim}
x = Distributed.@spawnat 2 begin
      sleep(3)
      rand(3)
    end
fetch(x) # espera y devuelve el resultado
\end{verbatim}
\end{frame}

\begin{frame}[label={sec:org36aa180},fragile]{@spawnat}
 \begin{verbatim}
function p()
  rand(3)
end
x = Distributed.@spawnat 2 p()
fetch(x) # error
\end{verbatim}
\end{frame}

\begin{frame}[label={sec:orgaca797f},fragile]{@spawnat}
 \begin{itemize}
\item Mover datos es una operación costosa.
\item Hay que evitar mover datos innecesariamente.
\end{itemize}

\begin{verbatim}
A = rand(1000, 1000)
x = @spawnat :any A^2

vs

x = @spawnat :any rand(1000, 1000)^2
\end{verbatim}
\end{frame}

\begin{frame}[label={sec:org4bc7df6},fragile]{WorkerPool}
 \begin{itemize}
\item Distribuir el trabajo manualmente mediante PIDs puede no ser práctico.
\item Podemos usar \texttt{WorkerPool} para administrar un subconjunto de nodos.
\item Por defecto ya hay un WorkerPool con todos los nodos creados.
\item Podemos usar \texttt{remotecall} para hacer llamadas.
\end{itemize}
\end{frame}

\begin{frame}[label={sec:org3c8a5b7},fragile]{WorkerPool}
 \begin{verbatim}
wp = WorkerPool([2, 3])
A = rand(1000)
f = remotecall(maximum, wp, A)
fetch(f)
\end{verbatim}

\begin{itemize}
\item Se elige un nodo que esté ocioso para ejecutar la función seleccionada, con los datos que enviamos.
\end{itemize}
\end{frame}

\begin{frame}[label={sec:orgecbceaa},fragile]{@everywhere}
 \begin{itemize}
\item Una opción para ejecutar algo en todos los nodos es usar \texttt{@everywhere}.
\item Útil para enviar las funciones y variables globales.
\end{itemize}
\end{frame}

\begin{frame}[label={sec:orgcb9dd1b},fragile]{@everywhere}
 \begin{verbatim}
@everywhere function sin_vect(A)
  sin.(A)
end

@spawnat :any begin
  A = rand(1000)
  sin_vect(A)
end
\end{verbatim}
\end{frame}

\begin{frame}[label={sec:orgb8a4279},fragile]{@distributed}
 \begin{itemize}
\item De forma similar a \texttt{@threads}, existe \texttt{@distributed} para paralelizar de forma sencilla bucles for.
\item Una limitación fundamental: al no haber memoria compartida no podemos modificar datos en vectores/matrices en paralelo.
\end{itemize}
\end{frame}

\begin{frame}[label={sec:org36b766a},fragile]{@distributed}
 \begin{verbatim}
A = rand(1000)
B = zeros(size(A))

Threads.@threads for i in 1:length(A)
  B[i] = sqrt(A[i])
end

Distributed.@distributed for i in 1:length(A)
  B[i] = sqrt(A[i])
end
\end{verbatim}
\end{frame}

\begin{frame}[label={sec:orgd3e0d89}]{SharedArrays}
\begin{itemize}
\item Podemos solventar esta limitación mediante los SharedArrays.
\item Replicamos el comportamiento de la memoria compartida, también con sus defectos (data races).
\item Limitación: necesita memoria compartida. Solo funciona si los nodos son todos locales.
\end{itemize}
\end{frame}

\begin{frame}[label={sec:org3fd9e9e}]{DistributedArrays}
\begin{itemize}
\item Arrays distribuidos entre los nodos. Cada nodo tiene solo acceso a la parte que se le ha asignado del array.
\item Solo el proceso de control puede ver la imagen completa. Aunque eso no quiere decir que almacene una copia.
\item Por red solo se transmiten los trozos del array que necesita el nodo en concreto.
\end{itemize}
\end{frame}

\begin{frame}[label={sec:org9de50e1},fragile]{DistributedArrays}
 \begin{verbatim}
@everywhere begin
  import Pkg
  Pkg.add("DistributedArrays")
  using DistributedArrays
end
\end{verbatim}
\end{frame}

\begin{frame}[label={sec:orgbf48adf},fragile]{DArray}
 \begin{itemize}
\item Un DArray se distingue de un array normal por varios detalles:
\begin{itemize}
\item Se especifica en qué workers reside (especificado mediante PIDs)
\item Se especifica en cuantos chunks se tiene que dividir por cada dimensión de la matriz.
\begin{itemize}
\item Por ejemplo, una matriz 100x100 con dist 2,2 distribuirá la matriz en los siguientes grupos: (1:50, 1:50), (1:50, 51:100), (51:100, 1:50) y (51:100, 51:100).
\end{itemize}
\end{itemize}
\item La forma más fácil de crear un DArray es partir de un array normal y usar \texttt{distribute}.
\item Para acceder a la parte local de un DArray, usamos \texttt{[:L]}.
\end{itemize}
\end{frame}

\begin{frame}[label={sec:orgf807e5c},fragile]{DArray}
 \begin{verbatim}
A = rand(100, 100)
B = distribute(A, dist=(2, 2))

@sync for w in Distributed.workers()
    @spawnat w begin
        B[:L] = B[:L] .+ 1
    end
end
\end{verbatim}
\end{frame}

\begin{frame}[label={sec:org58d4920},fragile]{DArray}
 \begin{itemize}
\item Podemos obtener qué rango de índices del array grande tenemos en el worker mediante \texttt{localindices}.
\item Convertimos el DArray en array normal usando \texttt{Array}.
\item Todos los nodos tienen acceso de lectura. Solo un nodo, el propietario, tiene acceso de escritura al propio chunk.
\item Aun así, es interesante dividir los arrays de lectura, para poder enviar menos datos a los nodos.
\end{itemize}
\end{frame}
\end{document}
